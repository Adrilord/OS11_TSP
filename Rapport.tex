\documentclass[10pt,a4paper]{article}
\usepackage[utf8]{inputenc}
\usepackage[french]{babel}
\usepackage[T1]{fontenc}
\usepackage{amsmath}
\usepackage{amsthm}
\usepackage{amsfonts}
\usepackage{amssymb}
\usepackage{graphicx}
\usepackage{mathrsfs}
\usepackage{url}
\author{\textsc{TRAN Quoc Nhat Han} \& \textsc{Adriene WARTELLE}}
\title{Rapport de projet OS13\\Analyse de politique de maintenance}
\date{\today}
\begin{document}
\maketitle
\renewcommand{\contentsname}{Sommaire}
\tableofcontents
\begin{abstract}
Soient des données liées à la fonctionnement de système, nous déterminons un modèle approprié et puis choisir une politique de maintenance optimal.
\end{abstract}
\section{Maintenance basant sur l'âge}
\subsection{Rappel}
Considérons un système non maintenu. En l'observant, nous obtenons un liste des dates de panne, grâce auquel nous construirerons une politique de remplacement systématique basée sur l'âge : \emph{Nous remplaçons lorsque le système tombe en panne ou qu'il survit une durée $t_0$}.

Le but est de minimiser le coût moyen cumulé.
\begin{equation}
    \label{coutmoy}
    \mathbb{E}(C) = \frac{\mathbb{E}(C(S))}{\mathbb{E}(S)}
\end{equation}

Où $S$ est la variable aléatoire représentant la date de remplacement et $C(S)$ est le coût de maintenance cumulé à l'instant $S$ (sachant que $C(S)$ est $c_c$ si une maintenance corrective et $c_p$ si préventive).
\subsection{Modéliser la durée de vie du système}
\end{document}